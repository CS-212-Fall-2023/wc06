\documentclass[a4paper]{exam}

\usepackage{amsmath,amssymb, amsthm}
\usepackage[a4paper]{geometry}
\usepackage{hyperref}
\usepackage{mdframed}

\title{Weekly Challenge 06: Context-Free Languages}
\author{CS 212 Nature of Computation\\Habib University}
\date{Fall 2023}

\theoremstyle{theorem}
\newtheorem{theorem}{Theorem}

\theoremstyle{claim}
\newtheorem{claim}{Claim}

\qformat{{\large\bf \thequestion. \thequestiontitle}\hfill}
\boxedpoints

\usepackage{draftwatermark}
\SetWatermarkText{Sample Solution}
\SetWatermarkScale{3}
\printanswers

\begin{document}
\maketitle

\begin{questions}

\titledquestion{Closure}

  Given the following theorem, prove or disprove the given claim.

  \begin{theorem}
    The class of context-free languages is not closed under intersection.
  \end{theorem}
  \begin{claim}
    The class of context-free languages is not closed under complementation.
  \end{claim}

  \begin{solution}
    We prove that the claim is true using a proof by contradiction.

    \begin{proof}
      Assume that the class of context-free languages is closed under complementation.\\
      Let $L_1$ and $L_2$ be context-free languages.\\
      Then $L_1'$ and $L_2'$ are context-free languages.\\
      Then $L_1'\cup L_2'$ is a context-free language.\\
      Then $(L_1'\cup L_2')'$ is a context-free language.\\
      Then $L_1\cap L_2$ is a context-free language.\qquad \textcolor{red}{$\bot$}
    \end{proof}
  \end{solution}
  
\end{questions}
\end{document}

%%% Local Variables:
%%% mode: latex
%%% TeX-master: t
%%% End:
